%% Generated by Sphinx.
\def\sphinxdocclass{report}
\documentclass[letterpaper,10pt,english]{sphinxmanual}
\ifdefined\pdfpxdimen
   \let\sphinxpxdimen\pdfpxdimen\else\newdimen\sphinxpxdimen
\fi \sphinxpxdimen=.75bp\relax

\usepackage[utf8]{inputenc}
\ifdefined\DeclareUnicodeCharacter
 \ifdefined\DeclareUnicodeCharacterAsOptional
  \DeclareUnicodeCharacter{"00A0}{\nobreakspace}
  \DeclareUnicodeCharacter{"2500}{\sphinxunichar{2500}}
  \DeclareUnicodeCharacter{"2502}{\sphinxunichar{2502}}
  \DeclareUnicodeCharacter{"2514}{\sphinxunichar{2514}}
  \DeclareUnicodeCharacter{"251C}{\sphinxunichar{251C}}
  \DeclareUnicodeCharacter{"2572}{\textbackslash}
 \else
  \DeclareUnicodeCharacter{00A0}{\nobreakspace}
  \DeclareUnicodeCharacter{2500}{\sphinxunichar{2500}}
  \DeclareUnicodeCharacter{2502}{\sphinxunichar{2502}}
  \DeclareUnicodeCharacter{2514}{\sphinxunichar{2514}}
  \DeclareUnicodeCharacter{251C}{\sphinxunichar{251C}}
  \DeclareUnicodeCharacter{2572}{\textbackslash}
 \fi
\fi
\usepackage{cmap}
\usepackage[T1]{fontenc}
\usepackage{amsmath,amssymb,amstext}
\usepackage{babel}
\usepackage{times}
\usepackage[Bjarne]{fncychap}
\usepackage[dontkeepoldnames]{sphinx}

\usepackage{geometry}

% Include hyperref last.
\usepackage{hyperref}
% Fix anchor placement for figures with captions.
\usepackage{hypcap}% it must be loaded after hyperref.
% Set up styles of URL: it should be placed after hyperref.
\urlstyle{same}
\addto\captionsenglish{\renewcommand{\contentsname}{Contents:}}

\addto\captionsenglish{\renewcommand{\figurename}{Fig.}}
\addto\captionsenglish{\renewcommand{\tablename}{Table}}
\addto\captionsenglish{\renewcommand{\literalblockname}{Listing}}

\addto\captionsenglish{\renewcommand{\literalblockcontinuedname}{continued from previous page}}
\addto\captionsenglish{\renewcommand{\literalblockcontinuesname}{continues on next page}}

\addto\extrasenglish{\def\pageautorefname{page}}

\setcounter{tocdepth}{1}



\title{Digit recognition Documentation}
\date{Dec 12, 2017}
\release{}
\author{Amandeep Sangeetha}
\newcommand{\sphinxlogo}{\vbox{}}
\renewcommand{\releasename}{Release}
\makeindex

\begin{document}

\maketitle
\sphinxtableofcontents
\phantomsection\label{\detokenize{index::doc}}



\chapter{classifiers package}
\label{\detokenize{classifiers:welcome-to-digit-recognition-s-documentation}}\label{\detokenize{classifiers:classifiers-package}}\label{\detokenize{classifiers::doc}}

\section{Submodules}
\label{\detokenize{classifiers:submodules}}

\section{classifiers.backpropagation module}
\label{\detokenize{classifiers:module-classifiers.backpropagation}}\label{\detokenize{classifiers:classifiers-backpropagation-module}}\index{classifiers.backpropagation (module)}\index{Backpropagation (class in classifiers.backpropagation)}

\begin{fulllineitems}
\phantomsection\label{\detokenize{classifiers:classifiers.backpropagation.Backpropagation}}\pysiglinewithargsret{\sphinxbfcode{class }\sphinxcode{classifiers.backpropagation.}\sphinxbfcode{Backpropagation}}{\emph{learning\_rate=3.5}, \emph{iterations=10}}{}
Bases: \sphinxcode{object}
\index{accuracy() (classifiers.backpropagation.Backpropagation method)}

\begin{fulllineitems}
\phantomsection\label{\detokenize{classifiers:classifiers.backpropagation.Backpropagation.accuracy}}\pysiglinewithargsret{\sphinxbfcode{accuracy}}{\emph{test\_contents}}{}
\end{fulllineitems}

\index{backpropagation\_algo() (classifiers.backpropagation.Backpropagation method)}

\begin{fulllineitems}
\phantomsection\label{\detokenize{classifiers:classifiers.backpropagation.Backpropagation.backpropagation_algo}}\pysiglinewithargsret{\sphinxbfcode{backpropagation\_algo}}{\emph{t\_image}, \emph{t\_ans}}{}
\end{fulllineitems}

\index{feature\_update() (classifiers.backpropagation.Backpropagation method)}

\begin{fulllineitems}
\phantomsection\label{\detokenize{classifiers:classifiers.backpropagation.Backpropagation.feature_update}}\pysiglinewithargsret{\sphinxbfcode{feature\_update}}{\emph{small\_tc}, \emph{update\_b}, \emph{update\_w}}{}
\end{fulllineitems}

\index{feed\_forward() (classifiers.backpropagation.Backpropagation method)}

\begin{fulllineitems}
\phantomsection\label{\detokenize{classifiers:classifiers.backpropagation.Backpropagation.feed_forward}}\pysiglinewithargsret{\sphinxbfcode{feed\_forward}}{\emph{image\_test}}{}
\end{fulllineitems}

\index{fill\_array() (classifiers.backpropagation.Backpropagation method)}

\begin{fulllineitems}
\phantomsection\label{\detokenize{classifiers:classifiers.backpropagation.Backpropagation.fill_array}}\pysiglinewithargsret{\sphinxbfcode{fill\_array}}{\emph{ans}}{}
\end{fulllineitems}

\index{get\_delta\_error() (classifiers.backpropagation.Backpropagation method)}

\begin{fulllineitems}
\phantomsection\label{\detokenize{classifiers:classifiers.backpropagation.Backpropagation.get_delta_error}}\pysiglinewithargsret{\sphinxbfcode{get\_delta\_error}}{\emph{small\_tc}, \emph{update\_b}, \emph{update\_w}}{}
\end{fulllineitems}

\index{gradient\_descent() (classifiers.backpropagation.Backpropagation method)}

\begin{fulllineitems}
\phantomsection\label{\detokenize{classifiers:classifiers.backpropagation.Backpropagation.gradient_descent}}\pysiglinewithargsret{\sphinxbfcode{gradient\_descent}}{}{}
\end{fulllineitems}

\index{load\_data() (classifiers.backpropagation.Backpropagation method)}

\begin{fulllineitems}
\phantomsection\label{\detokenize{classifiers:classifiers.backpropagation.Backpropagation.load_data}}\pysiglinewithargsret{\sphinxbfcode{load\_data}}{}{}
\end{fulllineitems}

\index{main() (classifiers.backpropagation.Backpropagation method)}

\begin{fulllineitems}
\phantomsection\label{\detokenize{classifiers:classifiers.backpropagation.Backpropagation.main}}\pysiglinewithargsret{\sphinxbfcode{main}}{\emph{layers}}{}
\end{fulllineitems}

\index{tc\_change\_error() (classifiers.backpropagation.Backpropagation method)}

\begin{fulllineitems}
\phantomsection\label{\detokenize{classifiers:classifiers.backpropagation.Backpropagation.tc_change_error}}\pysiglinewithargsret{\sphinxbfcode{tc\_change\_error}}{\emph{small\_tc}}{}
\end{fulllineitems}

\index{update\_derivate\_changes() (classifiers.backpropagation.Backpropagation method)}

\begin{fulllineitems}
\phantomsection\label{\detokenize{classifiers:classifiers.backpropagation.Backpropagation.update_derivate_changes}}\pysiglinewithargsret{\sphinxbfcode{update\_derivate\_changes}}{\emph{derivative}, \emph{changes}, \emph{change}}{}
\end{fulllineitems}


\end{fulllineitems}



\section{classifiers.knn module}
\label{\detokenize{classifiers:classifiers-knn-module}}\label{\detokenize{classifiers:module-classifiers.knn}}\index{classifiers.knn (module)}\index{KNN (class in classifiers.knn)}

\begin{fulllineitems}
\phantomsection\label{\detokenize{classifiers:classifiers.knn.KNN}}\pysiglinewithargsret{\sphinxbfcode{class }\sphinxcode{classifiers.knn.}\sphinxbfcode{KNN}}{\emph{X\_train}, \emph{Y\_train}, \emph{k=10}}{}
Bases: \sphinxcode{object}
\index{X\_train (classifiers.knn.KNN attribute)}

\begin{fulllineitems}
\phantomsection\label{\detokenize{classifiers:classifiers.knn.KNN.X_train}}\pysigline{\sphinxbfcode{X\_train}\sphinxbfcode{ = None}}
\end{fulllineitems}

\index{Y\_train (classifiers.knn.KNN attribute)}

\begin{fulllineitems}
\phantomsection\label{\detokenize{classifiers:classifiers.knn.KNN.Y_train}}\pysigline{\sphinxbfcode{Y\_train}\sphinxbfcode{ = None}}
\end{fulllineitems}

\index{get\_majority() (classifiers.knn.KNN static method)}

\begin{fulllineitems}
\phantomsection\label{\detokenize{classifiers:classifiers.knn.KNN.get_majority}}\pysiglinewithargsret{\sphinxbfcode{static }\sphinxbfcode{get\_majority}}{\emph{k\_nearest\_neighbors}}{}~\begin{quote}\begin{description}
\item[{Parameters}] \leavevmode
\sphinxstyleliteralstrong{k\_nearest\_neighbors} \textendash{} nearest neighbours

\item[{Returns}] \leavevmode
majority

\end{description}\end{quote}

\end{fulllineitems}

\index{get\_nearest\_neighbors() (classifiers.knn.KNN method)}

\begin{fulllineitems}
\phantomsection\label{\detokenize{classifiers:classifiers.knn.KNN.get_nearest_neighbors}}\pysiglinewithargsret{\sphinxbfcode{get\_nearest\_neighbors}}{\emph{x}}{}~\begin{quote}\begin{description}
\item[{Parameters}] \leavevmode
\sphinxstyleliteralstrong{x} \textendash{} single image array

\item[{Returns}] \leavevmode
nearest neighbors for given x

\item[{Return type}] \leavevmode
list

\end{description}\end{quote}

\end{fulllineitems}

\index{get\_prediction() (classifiers.knn.KNN method)}

\begin{fulllineitems}
\phantomsection\label{\detokenize{classifiers:classifiers.knn.KNN.get_prediction}}\pysiglinewithargsret{\sphinxbfcode{get\_prediction}}{\emph{x}}{}~\begin{quote}\begin{description}
\item[{Parameters}] \leavevmode
\sphinxstyleliteralstrong{x} \textendash{} single X for testing

\item[{Returns}] \leavevmode
Y predictin for a given test data

\end{description}\end{quote}

\end{fulllineitems}

\index{k (classifiers.knn.KNN attribute)}

\begin{fulllineitems}
\phantomsection\label{\detokenize{classifiers:classifiers.knn.KNN.k}}\pysigline{\sphinxbfcode{k}\sphinxbfcode{ = 10}}
\end{fulllineitems}

\index{predict() (classifiers.knn.KNN method)}

\begin{fulllineitems}
\phantomsection\label{\detokenize{classifiers:classifiers.knn.KNN.predict}}\pysiglinewithargsret{\sphinxbfcode{predict}}{\emph{X\_test}}{}~\begin{quote}\begin{description}
\item[{Parameters}] \leavevmode
\sphinxstyleliteralstrong{X\_test} (\sphinxstyleliteralemphasis{numpy array}) \textendash{} Test data of images

\item[{Returns}] \leavevmode
numpy array of labels for the given X

\end{description}\end{quote}

\end{fulllineitems}


\end{fulllineitems}



\section{classifiers.logistic\_regression module}
\label{\detokenize{classifiers:module-classifiers.logistic_regression}}\label{\detokenize{classifiers:classifiers-logistic-regression-module}}\index{classifiers.logistic\_regression (module)}\index{LogisticRegression (class in classifiers.logistic\_regression)}

\begin{fulllineitems}
\phantomsection\label{\detokenize{classifiers:classifiers.logistic_regression.LogisticRegression}}\pysiglinewithargsret{\sphinxbfcode{class }\sphinxcode{classifiers.logistic\_regression.}\sphinxbfcode{LogisticRegression}}{\emph{eta}}{}
Bases: \sphinxcode{object}
\index{fit() (classifiers.logistic\_regression.LogisticRegression method)}

\begin{fulllineitems}
\phantomsection\label{\detokenize{classifiers:classifiers.logistic_regression.LogisticRegression.fit}}\pysiglinewithargsret{\sphinxbfcode{fit}}{\emph{X}, \emph{y}, \emph{epoch=4000}}{}~\begin{quote}\begin{description}
\item[{Parameters}] \leavevmode\begin{itemize}
\item {} 
\sphinxstyleliteralstrong{X} \textendash{} X training images data

\item {} 
\sphinxstyleliteralstrong{y} \textendash{} training labels

\item {} 
\sphinxstyleliteralstrong{epoch} \textendash{} number of times to run (stopping criteria)

\end{itemize}

\end{description}\end{quote}

\end{fulllineitems}

\index{gradient\_decent() (classifiers.logistic\_regression.LogisticRegression method)}

\begin{fulllineitems}
\phantomsection\label{\detokenize{classifiers:classifiers.logistic_regression.LogisticRegression.gradient_decent}}\pysiglinewithargsret{\sphinxbfcode{gradient\_decent}}{\emph{X}, \emph{y}, \emph{y\_pred}}{}~\begin{quote}\begin{description}
\item[{Parameters}] \leavevmode\begin{itemize}
\item {} 
\sphinxstyleliteralstrong{X} \textendash{} training data

\item {} 
\sphinxstyleliteralstrong{y} \textendash{} y labels given

\item {} 
\sphinxstyleliteralstrong{y\_pred} \textendash{} y labels predicted

\end{itemize}

\end{description}\end{quote}

\end{fulllineitems}

\index{predict() (classifiers.logistic\_regression.LogisticRegression method)}

\begin{fulllineitems}
\phantomsection\label{\detokenize{classifiers:classifiers.logistic_regression.LogisticRegression.predict}}\pysiglinewithargsret{\sphinxbfcode{predict}}{\emph{X}}{}~\begin{quote}\begin{description}
\item[{Parameters}] \leavevmode
\sphinxstyleliteralstrong{X} \textendash{} test data

\item[{Returns}] \leavevmode
y labels for given X

\end{description}\end{quote}

\end{fulllineitems}


\end{fulllineitems}

\index{one\_to\_rest() (in module classifiers.logistic\_regression)}

\begin{fulllineitems}
\phantomsection\label{\detokenize{classifiers:classifiers.logistic_regression.one_to_rest}}\pysiglinewithargsret{\sphinxcode{classifiers.logistic\_regression.}\sphinxbfcode{one\_to\_rest}}{\emph{sample\_size=5000}, \emph{learning\_rate=0.1}}{}~\begin{quote}\begin{description}
\item[{Parameters}] \leavevmode\begin{itemize}
\item {} 
\sphinxstyleliteralstrong{sample\_size} \textendash{} sample of training data to include

\item {} 
\sphinxstyleliteralstrong{learning\_rate} \textendash{} learning rate for the logistic regression

\end{itemize}

\item[{Returns}] \leavevmode
accuracy of the algo

\end{description}\end{quote}

\end{fulllineitems}



\section{Module contents}
\label{\detokenize{classifiers:module-classifiers}}\label{\detokenize{classifiers:module-contents}}\index{classifiers (module)}

\chapter{experiments package}
\label{\detokenize{experiments:experiments-package}}\label{\detokenize{experiments::doc}}

\section{Subpackages}
\label{\detokenize{experiments:subpackages}}

\subsection{experiments.backpropogation package}
\label{\detokenize{experiments.backpropogation:experiments-backpropogation-package}}\label{\detokenize{experiments.backpropogation::doc}}

\subsubsection{Module contents}
\label{\detokenize{experiments.backpropogation:module-experiments.backpropogation}}\label{\detokenize{experiments.backpropogation:module-contents}}\index{experiments.backpropogation (module)}

\subsection{experiments.knn package}
\label{\detokenize{experiments.knn:experiments-knn-package}}\label{\detokenize{experiments.knn::doc}}

\subsubsection{Submodules}
\label{\detokenize{experiments.knn:submodules}}

\subsubsection{experiments.knn.k\_variation module}
\label{\detokenize{experiments.knn:experiments-knn-k-variation-module}}\label{\detokenize{experiments.knn:module-experiments.knn.k_variation}}\index{experiments.knn.k\_variation (module)}\index{k\_variation() (in module experiments.knn.k\_variation)}

\begin{fulllineitems}
\phantomsection\label{\detokenize{experiments.knn:experiments.knn.k_variation.k_variation}}\pysiglinewithargsret{\sphinxcode{experiments.knn.k\_variation.}\sphinxbfcode{k\_variation}}{}{}~\begin{quote}

To study the variation of K in KNN from 1 to 15
\end{quote}
\begin{quote}\begin{description}
\item[{Returns}] \leavevmode
creates csv with acc. and time taken

\end{description}\end{quote}

\end{fulllineitems}



\subsubsection{experiments.knn.run module}
\label{\detokenize{experiments.knn:module-experiments.knn.run}}\label{\detokenize{experiments.knn:experiments-knn-run-module}}\index{experiments.knn.run (module)}\index{run() (in module experiments.knn.run)}

\begin{fulllineitems}
\phantomsection\label{\detokenize{experiments.knn:experiments.knn.run.run}}\pysiglinewithargsret{\sphinxcode{experiments.knn.run.}\sphinxbfcode{run}}{}{}
To run Knn with sample size 5000
:return: prints the acc and time taken

\end{fulllineitems}



\subsubsection{experiments.knn.sample\_size\_variation module}
\label{\detokenize{experiments.knn:module-experiments.knn.sample_size_variation}}\label{\detokenize{experiments.knn:experiments-knn-sample-size-variation-module}}\index{experiments.knn.sample\_size\_variation (module)}\index{variate\_sample\_size() (in module experiments.knn.sample\_size\_variation)}

\begin{fulllineitems}
\phantomsection\label{\detokenize{experiments.knn:experiments.knn.sample_size_variation.variate_sample_size}}\pysiglinewithargsret{\sphinxcode{experiments.knn.sample\_size\_variation.}\sphinxbfcode{variate\_sample\_size}}{}{}
To study the variation of training sample size from 500 to MAX\_TRAIN\_DATA step size as 1000
:return: csv with acc. and time taken in each sample size

\end{fulllineitems}



\subsubsection{Module contents}
\label{\detokenize{experiments.knn:module-experiments.knn}}\label{\detokenize{experiments.knn:module-contents}}\index{experiments.knn (module)}

\subsection{experiments.logistic\_regresssion package}
\label{\detokenize{experiments.logistic_regresssion:experiments-logistic-regresssion-package}}\label{\detokenize{experiments.logistic_regresssion::doc}}

\subsubsection{Submodules}
\label{\detokenize{experiments.logistic_regresssion:submodules}}

\subsubsection{experiments.logistic\_regresssion.learning\_rate\_variation module}
\label{\detokenize{experiments.logistic_regresssion:module-experiments.logistic_regresssion.learning_rate_variation}}\label{\detokenize{experiments.logistic_regresssion:experiments-logistic-regresssion-learning-rate-variation-module}}\index{experiments.logistic\_regresssion.learning\_rate\_variation (module)}\index{learning\_rate\_variation() (in module experiments.logistic\_regresssion.learning\_rate\_variation)}

\begin{fulllineitems}
\phantomsection\label{\detokenize{experiments.logistic_regresssion:experiments.logistic_regresssion.learning_rate_variation.learning_rate_variation}}\pysiglinewithargsret{\sphinxcode{experiments.logistic\_regresssion.learning\_rate\_variation.}\sphinxbfcode{learning\_rate\_variation}}{}{}
To study the variation of learning rate in the logistic regression
the learning rate is varied from 0.1 to 0.9
:return:  creates a csv with time taken , accuracy

\end{fulllineitems}



\subsubsection{experiments.logistic\_regresssion.run module}
\label{\detokenize{experiments.logistic_regresssion:experiments-logistic-regresssion-run-module}}\label{\detokenize{experiments.logistic_regresssion:module-experiments.logistic_regresssion.run}}\index{experiments.logistic\_regresssion.run (module)}\index{run() (in module experiments.logistic\_regresssion.run)}

\begin{fulllineitems}
\phantomsection\label{\detokenize{experiments.logistic_regresssion:experiments.logistic_regresssion.run.run}}\pysiglinewithargsret{\sphinxcode{experiments.logistic\_regresssion.run.}\sphinxbfcode{run}}{}{}~\begin{quote}

To run the logistic regression with 5000 as sample size
\end{quote}
\begin{quote}\begin{description}
\item[{Returns}] \leavevmode
prints accuracy and time taken

\end{description}\end{quote}

\end{fulllineitems}



\subsubsection{experiments.logistic\_regresssion.sample\_size\_variation module}
\label{\detokenize{experiments.logistic_regresssion:experiments-logistic-regresssion-sample-size-variation-module}}\label{\detokenize{experiments.logistic_regresssion:module-experiments.logistic_regresssion.sample_size_variation}}\index{experiments.logistic\_regresssion.sample\_size\_variation (module)}\index{variate\_sample\_size() (in module experiments.logistic\_regresssion.sample\_size\_variation)}

\begin{fulllineitems}
\phantomsection\label{\detokenize{experiments.logistic_regresssion:experiments.logistic_regresssion.sample_size_variation.variate_sample_size}}\pysiglinewithargsret{\sphinxcode{experiments.logistic\_regresssion.sample\_size\_variation.}\sphinxbfcode{variate\_sample\_size}}{}{}
To study the variation of sample size in logistic regression
the variation are form 500 to MAX\_TRAIN\_DATA with step size of 1000
:return: creates a csv of observations

\end{fulllineitems}



\subsubsection{Module contents}
\label{\detokenize{experiments.logistic_regresssion:module-experiments.logistic_regresssion}}\label{\detokenize{experiments.logistic_regresssion:module-contents}}\index{experiments.logistic\_regresssion (module)}

\section{Module contents}
\label{\detokenize{experiments:module-contents}}\label{\detokenize{experiments:module-experiments}}\index{experiments (module)}

\chapter{Indices and tables}
\label{\detokenize{index:indices-and-tables}}\begin{itemize}
\item {} 
\DUrole{xref,std,std-ref}{genindex}

\item {} 
\DUrole{xref,std,std-ref}{modindex}

\item {} 
\DUrole{xref,std,std-ref}{search}

\end{itemize}


\renewcommand{\indexname}{Python Module Index}
\begin{sphinxtheindex}
\def\bigletter#1{{\Large\sffamily#1}\nopagebreak\vspace{1mm}}
\bigletter{c}
\item {\sphinxstyleindexentry{classifiers}}\sphinxstyleindexpageref{classifiers:\detokenize{module-classifiers}}
\item {\sphinxstyleindexentry{classifiers.backpropagation}}\sphinxstyleindexpageref{classifiers:\detokenize{module-classifiers.backpropagation}}
\item {\sphinxstyleindexentry{classifiers.knn}}\sphinxstyleindexpageref{classifiers:\detokenize{module-classifiers.knn}}
\item {\sphinxstyleindexentry{classifiers.logistic\_regression}}\sphinxstyleindexpageref{classifiers:\detokenize{module-classifiers.logistic_regression}}
\indexspace
\bigletter{e}
\item {\sphinxstyleindexentry{experiments}}\sphinxstyleindexpageref{experiments:\detokenize{module-experiments}}
\item {\sphinxstyleindexentry{experiments.backpropogation}}\sphinxstyleindexpageref{experiments.backpropogation:\detokenize{module-experiments.backpropogation}}
\item {\sphinxstyleindexentry{experiments.knn}}\sphinxstyleindexpageref{experiments.knn:\detokenize{module-experiments.knn}}
\item {\sphinxstyleindexentry{experiments.knn.k\_variation}}\sphinxstyleindexpageref{experiments.knn:\detokenize{module-experiments.knn.k_variation}}
\item {\sphinxstyleindexentry{experiments.knn.run}}\sphinxstyleindexpageref{experiments.knn:\detokenize{module-experiments.knn.run}}
\item {\sphinxstyleindexentry{experiments.knn.sample\_size\_variation}}\sphinxstyleindexpageref{experiments.knn:\detokenize{module-experiments.knn.sample_size_variation}}
\item {\sphinxstyleindexentry{experiments.logistic\_regresssion}}\sphinxstyleindexpageref{experiments.logistic_regresssion:\detokenize{module-experiments.logistic_regresssion}}
\item {\sphinxstyleindexentry{experiments.logistic\_regresssion.learning\_rate\_variation}}\sphinxstyleindexpageref{experiments.logistic_regresssion:\detokenize{module-experiments.logistic_regresssion.learning_rate_variation}}
\item {\sphinxstyleindexentry{experiments.logistic\_regresssion.run}}\sphinxstyleindexpageref{experiments.logistic_regresssion:\detokenize{module-experiments.logistic_regresssion.run}}
\item {\sphinxstyleindexentry{experiments.logistic\_regresssion.sample\_size\_variation}}\sphinxstyleindexpageref{experiments.logistic_regresssion:\detokenize{module-experiments.logistic_regresssion.sample_size_variation}}
\end{sphinxtheindex}

\renewcommand{\indexname}{Index}
\printindex
\end{document}